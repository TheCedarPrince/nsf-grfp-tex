%! TeX program = lualatex

\documentclass[11pt]{extarticle}

% Set 1-inch margins
\usepackage[margin=1in]{geometry}

% Set Times New Roman for text and Cambria Math for math fonts
\usepackage{fontspec}
\setmainfont{Times New Roman}

% Use Symbol font for non-alphabetic characters
\usepackage{textcomp}

% Set line spacing to single (no less than single-spacing)
\usepackage{setspace}
\setstretch{1.0}

% Package for handling citations
\usepackage[backend=biber,style=numeric,sorting=none]{biblatex}
\addbibresource{references.bib}

% For inserting equations as images
\usepackage{graphicx}

% Begin document
\begin{document}

\textbf{Personal Background:} I love seeing how things connect and one of the most rewarding aspects of leading the JuliaHealth volunteer organization is the ability to make connections between people and their shared interests
Our ~300 member groups spans students just learning the basics of programming all the way to physician-scientists.
To be in the middle of this is exhilarating as I get to learn from all the members about what they are working on, what they aspire to do, and what could happen in the future.
But perhaps most special of all is how I can see someone working in one field, say geospatial analysis, and connect them to someone working on climate health.
Through these synergies, I get to watch new collaborations start and the community goes along with them to start a direction to benefit us all.

It was thinking about connections that led me to spend a part of my summer in 2019 doing medical anthropology research in the foothills of rural Guatemala.
Although I was pursuing (and subsequently finished in 2020) a BS in biomedical engineering at the time, something that was so crucial to me was not only focusing on technology but also remembering those around me.
As I conducted health systems strengthening research to see how resilient and strong a country's medical system could be, I visited state clinics, indigenous villages, and even witch doctors to get a full sense of a health system.
I couldn't help but be fascinated by the fine interplay that exists in setting up a health system for an entire country.

I didn't have to wait long (only a little over a year in late 2020) to investigate these relationships at a national scale as I later joined Georgia Tech Research Institute and the US Centers for Disease Control in a dual role to support the national policy response to the COVID19 pandemic.
In this rapidly evolving role, everyday, my team and I thought about questions such as "what might this data say about that", "did this policy cause a change over there?", or, and the one most constantly on our minds, "is this having any effect?"
While in this position that I kept coming back to my own uneasy thought -- "what connections are we missing here? Surely, there must be a better way to find out?"

And it was through this struggle to find a more generic or abstract way of thinking around connections that lead me to mathematics.
In mathematics, these relationships that I was so obsessed with would be better understood as functions or morphisms and the connections I loved to make, to create something novel out of separate things, would be known as composition.
It was in encountering these ideas that made me realize I had reached a wall as an engineer.
By serendipity, I was offered at that moment a full scholarship to study and obtain a Master's at Northeastern University.
With a heartfelt goodbye to my colleagues, I joined Northeastern in the fall of 2023 with one goal: to study Applied Mathematics. 
To me, it felt like the best way to not leave behind my heritage of applied sciences like engineering while also having the freedom to explore how "things connect together" in a mathematically rigorous way.

After I complete my MS, I will then continue PhD studies in Applied Mathematics with a focus on the emerging and strongly intersectional field of applied category theory.
My goal will be to explore how a category theoretic framing of problems can be applied when one wants to relate seemingly disparate ideas across various problem spaces.
With this approach rooted in mathematics, I will explore the general theoretic limits of applied category theory alongside the Topos Institute and GATAS Lab (located at the University of Florida).
To examine specific applications of category theory, I will collaborate with the MIT Zardini Lab on engineering design problems and pursue a PhD with Dr. Nathaniel Osgood to explore broadly the treatment of dynamical systems modeling with category.
Through the NSF GRFP, I will pursue these research objectives and explore how I can translate these findings to various volunteer and open science-focused organizations I belong to.

\textbf{Intellectual Merit:} Initially, I was strongly encouraged to explore how "things connect together" by my small community college professor and mentor, Dr. Sri Kamesh Narasimhan at Corning Community College.
At the time, I was doing lab-based material science research on doping carbon with silver nanoparticles for the goal of creating a low-cost but effective water filtration system that could be deployed in resource constrained regions lacking water sanitation facilities.
Dr. Narasimhan saw the potential not just in my work but also in myself as a first-generation college student from rural Upstate New York in becoming a researcher.
I am so thankful for his recommendations as it was through his support, that I formed a small research team around this work and applied to the 2017 NSF Community College Innovation Challenge.
Our project, Project WaterFED (which stood for Water Filtration and Economic Development), was selected by the NSF as finalists and were invited to Washington, DC to meet with NSF I-Corps leaders and present our work to senators and national scientists at the US Capitol Hill.
I learned several things from that experience: 1) That to impact change on a community level, I need to think much more broadly, 2) There were more opportunities to enact positive action in the computational sciences, 3) The idea of public health and supporting a community encompasses so much more than I originally thought.
It was during this visit to Capitol Hill that I learned I had been accepted to Georgia Institute of Technology (Georgia Tech) to pursue a BS in Biomedical Engineering and I knew that to gain this broader perspective, I had to shift my focus from material science to computational science.

Equipped with these lessons, I taught myself much of the basics of programming before arriving at Georgia Tech, ready to learn how to approach problems much more broadly.
Within my first month at Georgia Tech, I found what I considered to be the ideal research position at Emory University's Department of Biomedical Informatics (Emory DBMI) as it blended biomedical informatics, public health, and engineering in one place.
I joined the Clifford Lab as a Data Information Specialist under the leadership of Gari Clifford, DPhil, the chair of Emory DBMI,  where I became the co-lead of a project that studied aging-in-place of geriatric populations suffering from cognitive impairment.
There, we used techniques from ubiquitous computing to monitor movement patterns of persons at their home to assess progression of cognitive impairment and what might environmental interventions look like.
Our work eventually led to us filing a patent (No. 62/811,266) and helped form the basis for the joint Emory/Georgia Tech Cognitive Empowerment Program which involved over 70 stakeholders from Emory Healthcare, Georgia Tech Research Institute (GTRI), and several research groups.

Although this effort exposed me to many uses of computing for social contexts (e.g. ubiquitous computing, data science, and resource constrained development), I was still eager to continue broadening my perspectives.
Through Dr. Clifford, I was introduced to Dr. Hall-Clifford, the chair for the Department of Anthropology at Agnes Scott College, and was invited to attend the National Association for the Practice of Anthropology (NAPA-OT) field school in rural Guatemala.
I was awarded a scholarship to attend this field school from Georgia Tech and spent part of my 2019 summer in Antigua where I investigated health system strengthening and what is involved in running a country's health system.
We conducted numerous field interviews with nurses, clinicians, and patients to learn more about the health system and that fully opened my eyes to the complexity of the intertwining relationships involved with the operation of providing health for all.
Our work led to a publication on the strength of the Guatemalan health system but also succeeded in broadening my perspectives even further to grapple with how to understand problems that can affect an entire nation.

After graduating from Georgia Tech, I then was hired into a dual role at GTRI and the US Centers for Disease Control (CDC) where my number one priority was to focus on the national COVID pandemic response through support of the CDC Office of Science.
One of my most major projects was involved in tracing the mobility patterns of US citizens and determining whether or not we could ascertain any underlying relationships between movement, reaction to CDC policies, and the spread (or lack thereof) of COVID cases.
Our work was formally recognized by the Georgia governor's office in our effort to stem the spread of the virus across the state of Georgia and the country.

Beyond my role in monitoring the COVID19 pandemic, I was also involved in several other projects related to national surveillance and public health.
One endeavor was collaborating with the Behavioral Risk Factor Surveillance System team to assess how feasible it would be to use electronic health records to ascertain national trends.
Using half a million patient medical records, we were adequately able to show that this sort of data could be used alongside national survey data to give greater insight into very specific as well as understudied populations.

Beyond working to understand the role of national policy in the COVID pandemic, one of the other most influential projects to me was being the principal investigator on a transnational study investigating disparities within mental health care.
After having been awarded approximately \$80,000 worth of grant money, I managed a small team of undergraduate research assistants to investigate chronic mental illness across the state of Georgia (~10 million patients).
This work earned me a nomination for an OHDSI Titan (the highest award given to a researcher in the field of observational health research) and my team was awarded the best community contribution award within observational health research.
Leveraging this momentum, I expanded the study and began collaborating with researchers from Asia and Europe to expand this study to ~150 million patients worldwide -- a feat only possible due to modernization advances in the observational health research space.

Within this role at GTRI and the CDC, I finally felt like I was at the level of perspective I needed to approach community problems but felt a malease about my approaches -- they felt too mechanistic and time-consuming. 
I had the perspective I had sought after, but I lacked the right tools to leverage all the connections I could now see from my perch -- from one single person all the way up to national scales.
While contemplating this problem, I was pulled onto a project that involved pure mathematics.
It was here I was exposed to category theory for the first time and it was so obvious to me the solution I needed to be able to work at the level of abstraction I was thinking about was right in front of me, in this field of mathematics.
I started teaching myself some of the fundamentals of pure mathematics on my own time and became completely convinced that this was in fact the area I should focus on next.

As luck would have it, Northeastern University (NEU) took notice of my work in observational health research.
They offered me an affiliate role at the NEU Roux Institute, an institute dedicated to bringing cutting edge health research methods along the New England corridor, and a full scholarship to study a field of my choice at NEU.
I jumped on the opportunity and chose to pursue an MS in Applied Mathematics.
In this role, I have been able to study category theory and other abstract maths directly while participating and conducting practical research in observational health.
\textbf{Being able to have research experience at multiple different levels, from one single person to all the way across countries, makes me uniquely equipped to explore the connections present across different domains and how they may be combined together.}

\textbf{Broader Impacts:} In my role as a leader in the JuliaHealth community, I have taken an active role in promoting interdiscplinary research software development. Our group has more than $250$ members  across various research communities such as JuliaHealth 

\textbf{Future Goals:} 


\end{document}
