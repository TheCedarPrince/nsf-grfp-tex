%! TeX program = lualatex

\documentclass[11pt]{extarticle}

% Set 1-inch margins
\usepackage[margin=1in]{geometry}

% Set Times New Roman for text and Cambria Math for math fonts
\usepackage{fontspec}
\setmainfont{Times New Roman}

% Use Symbol font for non-alphabetic characters
\usepackage{textcomp}

% Set line spacing to single (no less than single-spacing)
\usepackage{setspace}
\setstretch{1.0}

% Package for handling citations
\usepackage[backend=biber,style=numeric,sorting=none]{biblatex}
\addbibresource{references.bib}

% For inserting equations as images
\usepackage{graphicx}

% Begin document
\begin{document}

\textbf{Personal Background:} Finding connections between ideas, things, or each other, is one of the most unique traits about what makes us human.
For me, this has been the defining constant in my academic and professional career.
Whether it has been through volunteer organizations that I help lead, such as JuliaHealth or Google Summer of Code, 
For me, this is one of the most rewarding aspects about leading the JuliaHealth volunteer organization as I get to see connections form between students to clinicians over their shared interests.
Even more exhilarating is making those connections occur between members based on their career aspirations and goals.
Through these synergies, I get to watch new collaborations start and the community goes along with them to start a direction to benefit us all.

It was thinking about connections that led me to spend a part of my summer in 2019 doing medical anthropology research in the foothills of rural Guatemala.
Although I was pursuing (and subsequently finished in 2020) a BS in biomedical engineering at the time, something that was so crucial to me was not only focusing on technology but also remembering those around me.
As I conducted health systems strengthening research to see how resilient and strong a country's medical system could be, I visited state clinics, indigenous villages, and even witch doctors to get a full sense of a health system.
I couldn't help but be fascinated by the fine interplay that exists in setting up a health system for an entire country.

I didn't have to wait long (only a little over a year in late 2020) to investigate these relationships at a national scale as I later joined Georgia Tech Research Institute and the US Centers for Disease Control in a dual role to support the national policy response to the COVID19 pandemic.
In this rapidly evolving role, everyday, my team and I thought about questions such as "what might this data say about that", "did this policy cause a change over there?", or, and the one most constantly on our minds, "is this having any effect?"
While in this position that I kept coming back to my own uneasy thought -- "what connections are we missing here? Surely, there must be a better way to find out?"

And it was through this struggle to find a more generic or abstract way of thinking around connections that lead me to mathematics.
In mathematics, these relationships that I was so obsessed with would be better understood as functions or morphisms and the connections I loved to make, to create something novel out of separate things, would be known as composition.
It was in encountering these ideas that made me realize I had reached a wall as an engineer.
By serendipity, I was offered at that moment a full scholarship to study and obtain a Master's at Northeastern University.
With a heartfelt goodbye to my colleagues, I joined Northeastern in the fall of 2023 with one goal: to study Applied Mathematics. 
To me, it felt like the best way to not leave behind my heritage of applied sciences like engineering while also having the freedom to explore how "things connect together" in a mathematically rigorous way.

After I complete my MS, I will then continue PhD studies in Applied Mathematics with a focus on the emerging and strongly intersectional field of applied category theory.
My goal will be to explore how a category theoretic framing of problems can be applied when one wants to relate seemingly disparate ideas across various problem spaces.
With this approach rooted in mathematics, I will explore the general theoretic limits of applied category theory alongside the Topos Institute and GATAS Lab (located at the University of Florida).
To examine specific applications of category theory, I will collaborate with the MIT Zardini Lab on engineering design problems and pursue a PhD with Dr. Nathaniel Osgood to explore broadly the treatment of dynamical systems modeling with category.
Through the NSF GRFP, I will pursue these research objectives and explore how I can translate these findings to various volunteer and open science-focused organizations I belong to.

\textbf{Intellectual Merit:} In 2015, I started my research career at Corning Community College investigating how to cheaply dope carbon with silver nanoparticles to create low-cost yet effective water filtration for resource constrained communities lacking water sanitation.
My mentor, Prof. Narasimhan, saw the potential not just in my work but also in myself as a first-generation college student from rural Upstate New York.
It was through Dr. Narasimhan's encouragement, that I formed a small research team based on this work and applied to the 2017 NSF Community College Innovation Challenge.
Our project, Project WaterFED (which stood for Water Filtration and Economic Development), was selected as a finalist in the challenge and we were invited to Washington DC to meet with leaders of the NSF I-Corps and to present our work to US senators and NSF scientists at Capitol Hill. 
I came away with two new professional goals: to broaden my perspective on how communities relate to one another and to shift my research career to computational science.

With an acceptance letter to Georgia Tech to study biomedical engineering in hand and these lessons on my mind, I started my studies and found the ideal research position at the Clifford Lab at Emory University led by Gari Clifford, DPhil. 
I was immediately introduced to a variety of discplines and their applications such as ubiquitous computing for monitoring progression of cognitive impairment of persons at home and mobile health to assess pregnancy complications in rural communities.
Moreover, this was my first encounter with using computational methods for knowledge discovery (i.e. deriving insights into a topic from data).
A lesson that has always stayed with me since working with Dr. Clifford was his growing concern in knowledge discovery for research in general that: "the biggest problem will not be a lack of data, but rather, how do we separate the meaningful 'signal' from the 'noise' given by too much data?"

While these experiences and discussions certainly appealed to my curiosity about computational science, I still felt the urgency to understand how these methods and approaches could be applied within communities across the globe.
Dr. Clifford saw my desire to understand the sociotechnical dynamics existing between theory and practical application and introduced me to Prof. Rachel Hall-Clifford, the head of the NAPA-OT Field School, a Guatemala-based summer school dedicated to the understanding and practice of medical anthropology.
Dr. Hall-Clifford invited me to spend part of my 2019 summer at the school to investigate how the Guatemalan healthcare system operated across the country and how it could potentially be strengthened.
Georgia Tech awarded me a travel scholarship to attend this school and I found myself visiting various healthcare and indigenous communities across Guatemala.
There, I conducted numerous interviews with nurses, clinicians, and patients to learn more about how people interact with and receive care from Guatemalan healthcare providers.
This experience was eye-opening as I learned from interviews the complexity of the intertwining relationships involved with  meeting the health for all a country's citizens.

By the time I graduated in 2020 with my BS, I felt I had accomplished my two professional goals and was ready for my new researcher dual role at Georgia Tech Research Institute (GTRI) and the Centers for Disease Control (CDC) in 2021 where my number one priority was to supporting the US COVID19 pandemic response.
With the emergence of variants, my first major research project encompassed analyzing and uncovering any mobility patterns of US citizens from aggregated mobile phone data.
The goal of this was to work with the COVID Mitigation Policy Analysis Unit to determine if we could observe any potential relationships existing between movement, the implementation of CDC spread prevention policies, and the number of reported COVID cases and COVID-related deaths.
The knowledge discovery process involved in these endeavors was intensive with results needing to be returned nearly daily (sometimes multiple times in a day) and multiple data sets having to be manually harmonized.
We were able to streamline some of these processes but with the evolving nature of the pandemic, any process improvements only made us realize how much faster we needed to be.
By the start of 2022, our work had slowed considerably on this effort and we were formally recognized by the state of Georgia governor's office for our effort to stem the spread of COVID across Georgia and the country.

Through these opportunities, I have been able to explore at multiple levels the variety of ways communities, disciplines, and methods can be related to one another.
The perspectives I have gained through my research experiences with Project WaterFED and at GTRI and the CDC has exposed me to a huge variety of general research and computational methods and how knowledge discovery is conducted through them.
Additionally, through Emory and the CDC, I have become equipped with a more circumspect perspective on the importance of accounting for the social aspects involved both with computational methods and in knowledge discovery processes.
\textbf{These experiences and techniques I have learned make me uniquely equipped to undertake the generalization of knowledge discovery I want to investigate using the language of category theory.}

\textbf{Broader Impacts:} When the COVID19 pandemic broke out, as a native of Upstate New York and a freshly minted biomedical engineer, I eagerly jumped into emergency volunteering in NY.
Working with The Center for Discovery, a non-profit dedicated to the care of children with severe cognitive impairment or disorders, I worked with their engineering department to develop a COVID spread surveillance system and dashboard across their 400+ acre campus that housed several hundred children.
This dashboard was in use for several months at the Center and during that time, not a single child was reported to have passed.

Additionally, during the pandemic, I started working on a mathematics animation engine called Javis.jl with one of my friends, Ole Kroger, within the Julia programming language.
This project has won much acclaim where it has been used by individuals and across multiple universities and institutions for creating animations.
Furthermore Ole and I presented this engine at the 2021 JuliaCon where our recording has been viewed nearly 10,000 times.
Finally, Javis.jl was highlighted by mathematician, educator, and YouTuber Grant Sanderson (aka 3blue1brown) in his Summer of Math Exposition series encouraging people from across the globe to make math teaching videos -- that video has been viewed nearly 750,000 times and has driven the successful adoption of Javis.jl even further.

Furthermore, while at GTRI, I was highly inspired by the Observational Health Data Science and Informatics (OHDSI) community in having a space to develop research software within observational health.
However, I thought that we could drive innovation even further within the context of the JuliaHealth organization (where I was already a member).
JuliaHealth was largely defunct at the time and I stepped in as leader of the organization to revitalize the community.
Since leading the organization, our group has grown to more than 300 members across various health-related research disciplines ranging anywhere from geospatial health informatics to medical imagining.
Within the organization, I organize monthly meetings that are open to the public where we have guest speakers and attendees ranging anywhere from students and hobbyists to clinicians and professors who discuss what they are researching or tools they are developing.
Cumulatively, across our videos and blog that we maintain, we have garnered over 5,000 views and attracted and retained many new contributors to the ecosystem.

Finally, for the past 4 years, I have directly mentored several fellows as part of Google Summer of Code (GSoC) and have acted as the Co-Admin for the Julia Language's participation within GSoC.
Many of my fellows have been able to learn best practices in software engineering and contribute to a variety of open source software.
Moreover, within JuliaHealth, we have been able to attract record numbers of contributors across medical imaging, magnetic resonance imaging, and even census analyses.
Additionally, as a Co-Admin, I have overseen between 65 - 75 mentors and mentees within the Julia community.
My Co-Admin and I have been able to successfully oversee dozens of fellow projects as well as provide support to mentors in the course of the GSoC fellowship period.

\textbf{Future Goals:} My goal in pursuing a PhD in Applied Mathematics is to further understanding in how category theoretic approaches could be used across multiple domains.
In working with Dr. Nathaniel Osgood, I will utilize category theory to provide novel perspectives on knowledge discovery and redefine the limits of current category theoretic machinery available for computational purposes through the development or improvement of open source research software.
After completing my PhD studies, I would like to go on to join either a research institute such as rejoining GTRI or the Topos Institute 

\end{document}
