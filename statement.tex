%! TeX program = lualatex

\documentclass[11pt]{extarticle}

% Set 1-inch margins
\usepackage[margin=1in]{geometry}

% Set Times New Roman for text and Cambria Math for math fonts
\usepackage{fontspec}
\setmainfont{Times New Roman}

% Use Symbol font for non-alphabetic characters
\usepackage{textcomp}

% Set line spacing to single (no less than single-spacing)
\usepackage{setspace}
\setstretch{1.0}

% Package for handling citations
\usepackage[backend=biber,style=numeric,sorting=none]{biblatex}
\addbibresource{references.bib}

% For inserting equations as images
\usepackage{graphicx}

% Begin document
\begin{document}

\textbf{Personal Background:} I love seeing how things connect and one of the most rewarding aspects of leading the JuliaHealth volunteer organization is the ability to make connections between people and their shared interests
Our ~300 member groups spans students just learning the basics of programming all the way to physician-scientists.
To be in the middle of this is exhilarating as I get to learn from all the members about what they are working on, what they aspire to do, and what could happen in the future.
But perhaps most special of all is how I can see someone working in one field, say geospatial analysis, and connect them to someone working on climate health.
Through these synergies, I get to watch new collaborations start and the community goes along with them to start a direction to benefit us all.

It was thinking about connections that led me to spend a part of my summer in 2019 doing medical anthropology research in the foothills of rural Guatemala.
Although I was pursuing (and subsequently finished in 2020) a BS in biomedical engineering at the time, something that was so crucial to me was not only focusing on technology but also remembering those around me.
As I conducted health systems strengthening research to see how resilient and strong a country's medical system could be, I visited state clinics, indigenous villages, and even witch doctors to get a full sense of a health system.
I couldn't help but be fascinated by the fine interplay that exists in setting up a health system for an entire country.

I didn't have to wait long (only a little over a year in late 2020) to investigate these relationships at a national scale as I later joined Georgia Tech Research Institute and the US Centers for Disease Control in a dual role to support the national policy response to the COVID19 pandemic.
In this rapidly evolving role, everyday, my team and I thought about questions such as "what might this data say about that", "did this policy cause a change over there?", or, and the one most constantly on our minds, "is this having any effect?"
While in this position that I kept coming back to my own uneasy thought -- "what connections are we missing here? Surely, there must be a better way to find out?"

And it was through this struggle to find a more generic or abstract way of thinking around connections that lead me to mathematics.
In mathematics, these relationships that I was so obsessed with would be better understood as functions or morphisms and the connections I loved to make, to create something novel out of separate things, would be known as composition.
It was in encountering these ideas that made me realize I had reached a wall as an engineer.
By serendipity, I was offered at that moment a full scholarship to study and obtain a Master's at Northeastern University.
With a heartfelt goodbye to my colleagues, I joined Northeastern in the fall of 2023 with one goal: to study Applied Mathematics. 
To me, it felt like the best way to not leave behind my heritage of applied sciences like engineering while also having the freedom to explore how "things connect together" in a mathematically rigorous way.

\textbf{Intellectual Merit:}

\textbf{Broader Impacts:} In my role as a leader in the JuliaHealth community, I have taken an active role in promoting interdiscplinary research software development. Our group has more than $250$ members  across various research communities such as JuliaHealth 


\end{document}
